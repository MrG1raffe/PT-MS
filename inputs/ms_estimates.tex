\documentclass[../TV&MS.tex]{subfiles}
\begin{document}
    
\section{Параметрическая статистика. Оценки}

\subsection{Определения}

Сначала вспомним, что есть такая штука, как статистическая структура
$\left( \Real^{n}, \Bor^{n}, \mathcal{P} \right)$.
Теперь немного уточним, что представляет собой $\mathcal{P}$.
В точечных оценках и в доверительных интервалах мы будем считать,
что $\mathcal{P} = \Pro_\theta$.
То есть мы считаем, что наша семейство распределений параметризовано каким-то векторным параметром $\theta \in \Theta \subset \Real^{m}$.
Например, $\Pro_\theta = \Norm\left( \theta_1, \theta_2^2 \right)$,
тогда пространство параметров $\Theta = (-\infty, \infty) \times (0, \infty)$.
Теперь, собственно, каким образом мы будем это оценивать.

\begin{Def}\label{ms:e:def:stat}
    \mdef{Статистикой} называется любая измеримая вектор-функция от выборки, не зависящая от $\theta$.
    \[
        T(\Smpl) = \bigl( T_1(\Smpl),\ldots, T_n(\Smpl) \bigr)^{T} \colon \Real^{n} \mapsto \Real^{m}
    .\] 
\end{Def} 

\begin{Def}\label{ms:e:def:estimate}
    \mdef{Оценкой} называется статистика, множество значений которой совпадает с $\Theta$
\end{Def} 

\begin{Def}\label{ms:e:def:unbiased}
    Оценка называется \mdef{несмещённой}, если $\Expec T(\Smpl) = \theta$.
\end{Def} 

\begin{Def}\label{ms:e:def:consistent}
    Оценка называется \mdef{состоятельной}, если $T(\Smpl) \xrightarrow[n\rightarrow \infty]{\Pro_\theta} \theta \quad \forall \theta \in \Theta$.
    Оценка называется \mdef{сильно состоятельной}, если сходимость почти наверное.
\end{Def} 

Теперь надо примеров каких-нибудь накидать.

\subsection{Примеры}

\begin{Ex}
    В предыдущем парагафе мы ввели выборочное среднее (см. опр.~\ref{ms:m:def:sampleMean}).
    Пройдемся по каждому из этих 4-х пунктов, если положить $T(\Smpl) = \overline{X}$, а $\theta$ имеет смысл матожидания.
    \begin{enumerate}
        \item \textbf{Измеримость.} Да, выборочное среднее измеримо как композиция сложения и деления, а эти функции являются измеримыми.
        \item \textbf{Множество значений.} Тут мы можем выстрелить себе в ногу или в какое другое место и искусственно ввести ограничение на пространоство параметров, но зачем? 
            Если не вводить никаких дурацких ограничений (например, оцениваем МО выборки из нормального распределения, но говорим, что МО обязательно не меньше 300), то все ОК.
        \item \textbf{Несмещённость.} 
            \[
                \Expec \overline{X} = \frac{1}{n}\sum\limits_{i=1}^{n} \Expec X_i = \Expec X_i
            \] 
            Следовательно, оценка является несмещённой.
        \item \textbf{Состоятельность} сразу следует из ЗБЧ, а сильная состоятельность~---~из УЗБЧ.
    \end{enumerate} 
\end{Ex} 

\begin{Ex}
    Чуть менее подробно рассмотрим выборочную дисперсию $\sigma^2$ (см. опр.~\ref{ms:m:def:sampleVar}).
    $T(\Smpl) = S^2 = \frac{1}{n}\sum\limits_{i=1}^{n} (X_i - \overline{X})^2$.
    Как было показано ранее~\eqref{ms:m:eq:biasedVar}, $T(\Smpl)$ не является несмещённой оценкой, но является состоятельной и даже сильно состоятельной.
    Хочется не париться и сказать, что вот здесь~\eqref{ms:m:eq:biasedVar} $\frac{n-1}{n}\sigma^2 \rightarrow \sigma^2$, но не тут-то было.
    Там мы считали матожидание оценки, а нам нужна сходимость самой оценки.
    Тогда может к той сумме прикрутить УЗБЧ и всё получится?
    Снова нет, так как в УЗБЧ мы генерировали значения из одного распределения, а тут для разных $n$ случайная величина $\left( X_i - \overline{X} \right)^2$, возможно, будет распределена по-разному ($\overline{X}$ зависит от $n$).
    УЗБЧ тут применить можно, но не сразу, надо сначала немного повертеть буквами. 
    Из этой~\eqref{ms:m:eq:biasedVar} выкладки следует, что:
    \begin{multline}
        S^2 = \frac{1}{n}\sum\limits_{i=1}^{n} \left(X_i - a\right)^2 -
        \left(\overline{X} - a\right)^2
        = \frac{1}{n}\sum\limits_{i=1}^{n}\left( X_i - a \right)^2 -
        \left(\frac{1}{n}\sum\limits_{i=1}^{n} X_i - a \right)^2
        \xrightarrow[n\rightarrow \infty]{\text{п.н.}} \\
        \xrightarrow[n\rightarrow \infty]{\text{п.н.}}
        \Expec \left( X_i - a \right)^2 - (a - a)^2
        = \Disp X_i
    .\end{multline} 
    Как-то так.
    Из простейших свойств пределов (не, ну тут правда очевидно) следует,
    что исправленная (несмещённая) выборочная дисперсия также является состоятельной оценкой теоретической дисперсии.
\end{Ex} 

\begin{Ex}
    Последнее, что хотелось бы тут рассмотреть и поковырять~---~несмещённую, но не состоятельную оценку.
    Тут придётся натянуть сову на глобус и рассмотреть выборку из $\Norm(a, 1)$ и оценку матожидания $T(\Smpl) = \frac{n-1}{n}X_1 + \frac{1}{n(n-1)}\sum\limits_{i=2}^{n} X_i$.
    Докажем несмещённость:
    \begin{equation}
        \Expec T(\Smpl) = \frac{n-1}{n} \Expec X_1 + \frac{1}{n} \Expec X_i = 
        \frac{n-1}{n}a + \frac{1}{n}a = a
    .\end{equation}
    Снова воспользовались нашим любимым ЗБЧ. А теперь несостоятельность:
    \begin{multline}
        \Pro \bigl( \left\lvert T(\Smpl) - a \right\rvert < \epsilon \bigr)
        = \Pro\left( \left\lvert \frac{n-1}{n}(X_1 - a) +
        \frac{1}{n(n-1)}\sum\limits_{i=2}^{n} (X_i - a) \right\rvert<\epsilon\right) =\\
        = \Pro \bigl( \left\lvert X_i - a \right\rvert < \epsilon \bigr) = 
        \Phi(\epsilon) - \Phi (-\epsilon) \cancel{\xrightarrow[n\rightarrow \infty]{}} 0
    .\end{multline} 
    Мы воспользовалить тем, что $X_i$ одинаково распределены, а потом тем,
    что $(X_i - a) \sim \Norm(0, 1)$, и подставили соответствующую вероятность.
\end{Ex} 

\newpage

\end{document} 
