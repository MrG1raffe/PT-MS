\documentclass[../TV&MS.tex]{subfiles}
\begin{document}
    
\section{Вероятностные неравенства}
\begin{Lem}
Для любой неотрицательной неубывающей функции $g(x)$ выполнено неравенство 
$$\Pro(|\xi| > x) \leqslant \frac{\Expec g(|\xi|)}{g(x)}$$
\end{Lem}
\begin{Proof} \\
\begin{multline*}
    \Expec g(|\xi|) = \Expec g(|\xi|) \Ind(|\xi| \geqslant x) +
    \Expec g(|\xi|) \Ind(|\xi| < x) \geqslant \\
    \Expec g(|\xi|) \Ind(|\xi| \geqslant x) \geqslant 
    g(x)\Expec\Ind(|\xi| \geqslant x) = g(x)\Pro(|\xi| > x)
\end{multline*}
Здесь мы воспользовались представлением $1 = \Ind(A) + \Ind(\overline{A})$, затем неотрицательностью функции и, следовательно, ее матожидания. Далее использовалась монотонность функции, и в последнем переходе тождество $\Expec \Ind(A) = 1 \Pro(A) + 0 \Pro(\overline{A}) = \Pro(A)$
\end{Proof}

Из этой леммы следуют два полезных неравенства.

\begin{Th} [неравенство Маркова]
Для любой случайной величины $\xi$, имееющей конечное $\Expec|\xi|$, выполнено
$$\Pro(|\xi| > \varepsilon) \le \frac{\Expec|\xi|}{\varepsilon}$$
\end{Th}
\begin{Proof}\\
В неравенстве леммы возьмем $g(x) = x$.
\end{Proof}

\begin{Th} [неравенство Чебышева]
Для любой случайной величины $\xi$, имееющей конечный первый и второй момент, выполнено
$$\Pro(|\xi - \Expec\xi| > \varepsilon) \le \frac{\Disp\xi}{\varepsilon^2}$$
\end{Th}
\begin{Proof}\\
В неравенстве леммы возьмем $g(x) = x^2$.
\end{Proof}

\newpage


\end{document}
