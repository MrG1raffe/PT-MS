\documentclass[../TV&MS.tex]{subfiles}
\begin{document}
    
\section{Предельные теоремы}

\begin{Th} [Пуассон]
Пусть $\lambda = np$. Тогда при малых $p$ и больших $n$ можно использовать приближение
$$\Pro(\mu_n = k) = C_n^kp^kq^{n-k} \approx \frac{e^{-\lambda}\lambda^k}{k!}$$
\end{Th}
\begin{Proof}
Докажем индукцией по $k$. При $k = 0$
$$\Pro(\mu_n = 0) = (1-p)^n = (1 - \frac{\lambda}{n})^n \longrightarrow e^{-\lambda} \quad (n \to \infty)$$
$$\frac{\Pro(\mu_n = k)}{\Pro(\mu_n = k-1)} = \frac{n!}{k!(n-k)!}p^kq^{n-k} \frac{(k-1)!(n-k+1)!}{n!} \frac{1}{p^{k-1}q^{n-k+1}} = \frac{(n-k+1)p}{kq}$$
$$\Pro(\mu_n = k) = \frac{(n-k+1)p}{kq}\Pro(\mu_n = k-1) = \frac{(n-k+1)\frac{\lambda}n}{k(1-\frac{\lambda}n)}\Pro(\mu_n = k-1)$$

Воспользуемся предположением индукции для $k-1$.
$$\Pro(\mu_n = k) \to \frac{(n-k+1)\frac{\lambda}n}{k(1-\frac{\lambda}n)} \frac{e^{-\lambda}\lambda^{k-1}}{(k-1)!} \longrightarrow \frac{e^{-\lambda}\lambda^{k}}{k!} (n \to \infty)$$
\end{Proof}

Следующая оценка формализует <<малость>> $p$ и <<величину>> $n$:
$$\sup\limits_k |\Pro(\mu_n = k) - \frac{e^{-\lambda}\lambda^k}{k!}| \le 2np^2$$
Таким образом, зная $n, p$ всегда можно оценить сверху погрешность аппроксимации.\\
\begin{Why}
При очень большом числе испытаний ни один нормальный компьютер не способен вычислить  $ C_n^kp^kq^{n-k}$, а уж тем более сложить их. Данная теорема позволяет сводить вычисление таких сложных вещей к вычислению экспонент.
\end{Why}

Распределение величины $\xi$ такое, что $\forall k \in \Int_+\quad  \Pro(\xi = k) =\frac{e^{-\lambda}\lambda^k}{k!}$, называется \underline{распределением Пуассона} 
и обозначается $\Pois(\lambda)$

Одно из приложений распределения Пуассона --- это пуассоновские потоки. Пусть во времени происходят некоторые события, которые мы фиксируем. $\lambda$ --- интенсивность потока --- показывает среднее число событий за единицу времени. Тогда число произошедших событий на отрезке $[0, t]$ равно $\xi_t\colon \Pro(\xi_t = k) = \frac{e^{-\lambda t}(\lambda t)^k}{k!}$.

Свойства пуассоновского распределения. ($\xi \sim \Pois(\lambda)$)
\begin{enumerate}
	\item $\Expec\xi = \lambda$
	\item $\Disp\xi = \lambda$
	\item $\xi_i \sim \Pois(\lambda_i) \quad i = 1, \ldots, n \Rightarrow \Sum{i}{1}{n}\xi_i \sim \Pois(\Sum{i}{1}{n}\lambda_i)$
\end{enumerate}

\begin{St}
Пусть независимые случайные величины $\xi_i \sim \Pois(\lambda_i), \quad i = 1, 2$. Тогда условное распределение $\xi_1$ при условии $\xi_1 + \xi_2 = n$ имеет биномиальное распределение с параметрами $n, \frac{\lambda_1}{\lambda_1 + \lambda_2}$, то есть
$$\Pro(\xi_1 = k | \xi_1 + \xi_2 = n) = C_n^kp^k(1-p)^{n-k}, \quad p = \frac{\lambda_1}{\lambda_1 + \lambda_2}$$
\end{St}
\begin{Proof}
$$\Pro(\xi_1 = k | \xi_1 + \xi_2 = n) = \frac{\Pro(\xi_1 = k, \xi_1 + \xi_2 = n}{\Pro(\xi_1 + \xi+2 = n)} = \frac{\Pro(\xi_1 = k, \xi_2 = n - k}{\Pro(\xi_1 + \xi+2 = n)}=$$
В числителе воспользуемся независимостью $\xi_1, \xi_2$, а в знаменателе свойством 3 пуассоновского распределения:
$$=\frac{e^{-\lambda_1}\frac{\lambda_1^k}{k!}e^{-\lambda_2}\frac{\lambda_2^{n-k}}{(n-k)!}}{e^{-(\lambda_1+\lambda_2)}\frac{(\lambda_1+\lambda_2)^n}{n!}} = 
C_n^k(\frac{\lambda_1}{\lambda_1 + \lambda_2})^k(\frac{\lambda_2}{\lambda_1 + \lambda_2})^{n-k}$$
\end{Proof}

\begin{Th} [Муавра-Лапласа]
Пусть $\sqrt{npq}$ велико. Тогда
$$\Pro(\mu_n = k) = \Pro(\frac{\mu_n - np}{\sqrt{npq}} = \frac{k - np}{\sqrt{npq}}) = \frac{1}{\sqrt{2\pi}}e^{-\frac{x^2}2} + o(1), \quad x \equiv \frac{k - np}{\sqrt{npq}}$$
\end{Th}
\begin{Proof} не было и не должно быть.
\end{Proof}

Рассмотрим функции
$$\phi(x) = \frac{1}{\sqrt{2\pi}}e^{-\frac{x^2}2}$$
$$\Phi(x) = \int\limits_{-\infty}^x \frac{1}{\sqrt{2\pi}}e^{-\frac{u^2}2} du = \int\limits_{-\infty}^{x}\phi(u)du$$

Заметим, что 
$$\int\limits_{-\infty}^{+\infty} \phi(u)du = 1$$ 
Это верно, так как данный интеграл сводится заменой $t = \frac{x}{\sqrt{2}}$ к известному интегралу Пуассона.

Так как $\phi(x)$ неотрицательна, и интеграл от нее по всей прямой равен 1, она является плотностью некоторого абсолютно непрерывного распределения, которое называется \underline{стандартным нормальным распределением} и обозначается $N(0, 1)$.

Пусть $\xi \sim N(0,1)$.
$$\Expec\xi^k = \int\limits_{-\infty}^{+\infty}x^k\phi(x)dx$$
$$e^{h\xi}=1 + h\xi + \frac{(h\xi)^2}{2!} + \ldots$$
$$\Expec e^{h\xi} = 1 + h\Expec\xi + \frac{h^2}{2!}\Expec\xi^2 + \ldots \equiv \psi(h)$$
Функция $\psi(h)$ называется \underline{производящей функцией моментов}.
$$\psi(h) = \int\limits_{-\infty}^{+\infty}e^{hx}\phi(x)dx = \int\limits_{-\infty}^{+\infty}\frac{1}{\sqrt{2\pi}}e^{hx-\frac{x^2}2}dx = \frac{e^{\frac{h^2}2}}{\sqrt{2\pi}}\int\limits_{-\infty}^{+\infty}e^{-\frac{(x-h)^2}2}dx = e^{\frac{h^2}2}$$
$$e^{\frac{h^2}2} = 1 + \frac{h^2}2 + \frac1{2!}(\frac{h^2}2)^2 + \ldots$$
Приравнивая это равенство к предыдущему разложению получим, что все нечетные центральные моменты равны 0.
$$\Expec\xi^{2n-1} = 0$$
$$\frac{1}{n!2^n} = \frac{\Expec\xi^{2n}}{(2n)!} \Rightarrow \Expec\xi^{2n} = \frac{(2n)!}{n!2^n} = \frac{2n(2n-1)\ldots n \ldots 1}{n!2^n} = (2n-1)!!$$

Последняя формула позволяет очень быстро получить нужный момент, не считая интеграл по честям много раз.

\begin{Th}[Интегральная теорема Муавра-Лапласа]
Пусть $\sqrt{npq}$ велико. Тогда
$$\Pro(m_1 \le \nu_n \le m_2) = \Pro(\frac{m_1 - np}{\sqrt{npq}} \le \frac{\mu_n - np}{\sqrt{npq}} \le \frac{m_2 - np}{\sqrt{npq}}) \approx \int\limits_{x_1}^{x_2}\phi(x)dx, \quad x1 \equiv \frac{m_1 - np}{\sqrt{npq}}, x2 \equiv\frac{m_2 - np}{\sqrt{npq}}$$
\end{Th}

С данной теоремой связана <<Задача о докторе Споке>>. (Добавить ссылку)

Рассмотрим теперь случайную величину $\xi$ с геометрическим распределением. Поскольку $\Expec\xi = \frac1p$, 
$$p\xi=\frac{\xi}{\Expec\xi}$$

При стремлении вероятности успеха к нулю, номер первого успеха будет стремиться к бесконечности, как и матожидание $\xi$. Однако их отношение будет иметь конечный предел:
$$\forall x > 0 \quad \Pro(p\xi > x) = \Pro(\xi > \frac{\xi}p) = \Sum{k}{[\frac{x}{p}] + 1}{\infty}p(1-p)^{k-1} = \frac{p(1-p)^{[\frac{x}p]}}p = (1-p)^{[\frac{x}p]}$$ 
$$\lim\limits_{p \to 0} \Pro(p\xi > x) = \lim\limits_{p \to 0}(1-p)^{[\frac{x}p]} = e^{-x}$$
В последнем переходе от дробной части можно избавиться, так как она не вносит никакого вклада в предел.
Итак, получили, что
$$\lim\limits_{p \to 0} \Pro(p\xi < x) = 1 - e^{-x}$$

Это частный случай \underline{показательного} (экспоненциального) распределения. В общем случае оно выглядит так:
$$F(x) = 1 - e^{-\lambda x}$$
$$p(x) = \lambda e^{-\lambda x}$$
Матожидание случайной величины, распределенной показательно, имеет вид
$$\int\limits_0^{\infty} x\lambda e^{-\lambda x}dx = \frac1{\lambda}$$

Показательно распределение обладает интересным свойством: свойством отсутвия последействия (или отсутствия памяти). Предположим, что вы ждете автобус на 
остановке, а время между автобусами $\xi$ имеет показательное распределение. Тогда вероятность того, что вы прождете автобус еще $t$ никак не зависит от того, 
как долго ($\tau$) вы уже ждете. Формализуем это:
$$ \Pro(\xi > t + \tau | \xi > \tau) = \Pro(\xi > t)$$
По определению условной вероятности
$$\Pro(\xi > t + \tau | \xi > \tau) = \frac{\Pro(\xi > t + \tau, \xi > \tau)}{\Pro(\xi > \tau)} = \frac{\Pro(\xi > t + \tau)}{\Pro(\xi > \tau)} = \frac{e^{-\lambda(t+\tau)}}{e^{-\lambda \tau}} = e^{-\lambda t}$$

Показательное распределение является единственным обладающим таким свойством в классе абсолютно непрерывных распределений. В классе дискретных распределений таким свойством обладает только геометрическое распределение.

\newpage


\end{document}
